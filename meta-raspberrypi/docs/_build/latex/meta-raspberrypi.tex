% Generated by Sphinx.
\def\sphinxdocclass{report}
\documentclass[letterpaper,10pt,english]{sphinxmanual}
\usepackage[utf8]{inputenc}
\DeclareUnicodeCharacter{00A0}{\nobreakspace}
\usepackage{cmap}
\usepackage[T1]{fontenc}

\usepackage{babel}
\usepackage{times}
\usepackage[Bjarne]{fncychap}
\usepackage{longtable}
\usepackage{sphinx}
\usepackage{multirow}
\usepackage{eqparbox}


\addto\captionsenglish{\renewcommand{\figurename}{Fig. }}
\addto\captionsenglish{\renewcommand{\tablename}{Table }}
\SetupFloatingEnvironment{literal-block}{name=Listing }



\title{meta-raspberrypi Documentation}
\date{May 12, 2020}
\release{master}
\author{meta-raspberrypi contributors}
\newcommand{\sphinxlogo}{}
\renewcommand{\releasename}{Release}
\setcounter{tocdepth}{1}
\makeindex

\makeatletter
\def\PYG@reset{\let\PYG@it=\relax \let\PYG@bf=\relax%
    \let\PYG@ul=\relax \let\PYG@tc=\relax%
    \let\PYG@bc=\relax \let\PYG@ff=\relax}
\def\PYG@tok#1{\csname PYG@tok@#1\endcsname}
\def\PYG@toks#1+{\ifx\relax#1\empty\else%
    \PYG@tok{#1}\expandafter\PYG@toks\fi}
\def\PYG@do#1{\PYG@bc{\PYG@tc{\PYG@ul{%
    \PYG@it{\PYG@bf{\PYG@ff{#1}}}}}}}
\def\PYG#1#2{\PYG@reset\PYG@toks#1+\relax+\PYG@do{#2}}

\expandafter\def\csname PYG@tok@gd\endcsname{\def\PYG@tc##1{\textcolor[rgb]{0.63,0.00,0.00}{##1}}}
\expandafter\def\csname PYG@tok@gu\endcsname{\let\PYG@bf=\textbf\def\PYG@tc##1{\textcolor[rgb]{0.50,0.00,0.50}{##1}}}
\expandafter\def\csname PYG@tok@gt\endcsname{\def\PYG@tc##1{\textcolor[rgb]{0.00,0.27,0.87}{##1}}}
\expandafter\def\csname PYG@tok@gs\endcsname{\let\PYG@bf=\textbf}
\expandafter\def\csname PYG@tok@gr\endcsname{\def\PYG@tc##1{\textcolor[rgb]{1.00,0.00,0.00}{##1}}}
\expandafter\def\csname PYG@tok@cm\endcsname{\let\PYG@it=\textit\def\PYG@tc##1{\textcolor[rgb]{0.25,0.50,0.56}{##1}}}
\expandafter\def\csname PYG@tok@vg\endcsname{\def\PYG@tc##1{\textcolor[rgb]{0.73,0.38,0.84}{##1}}}
\expandafter\def\csname PYG@tok@vi\endcsname{\def\PYG@tc##1{\textcolor[rgb]{0.73,0.38,0.84}{##1}}}
\expandafter\def\csname PYG@tok@mh\endcsname{\def\PYG@tc##1{\textcolor[rgb]{0.13,0.50,0.31}{##1}}}
\expandafter\def\csname PYG@tok@cs\endcsname{\def\PYG@tc##1{\textcolor[rgb]{0.25,0.50,0.56}{##1}}\def\PYG@bc##1{\setlength{\fboxsep}{0pt}\colorbox[rgb]{1.00,0.94,0.94}{\strut ##1}}}
\expandafter\def\csname PYG@tok@ge\endcsname{\let\PYG@it=\textit}
\expandafter\def\csname PYG@tok@vc\endcsname{\def\PYG@tc##1{\textcolor[rgb]{0.73,0.38,0.84}{##1}}}
\expandafter\def\csname PYG@tok@il\endcsname{\def\PYG@tc##1{\textcolor[rgb]{0.13,0.50,0.31}{##1}}}
\expandafter\def\csname PYG@tok@go\endcsname{\def\PYG@tc##1{\textcolor[rgb]{0.20,0.20,0.20}{##1}}}
\expandafter\def\csname PYG@tok@cp\endcsname{\def\PYG@tc##1{\textcolor[rgb]{0.00,0.44,0.13}{##1}}}
\expandafter\def\csname PYG@tok@gi\endcsname{\def\PYG@tc##1{\textcolor[rgb]{0.00,0.63,0.00}{##1}}}
\expandafter\def\csname PYG@tok@gh\endcsname{\let\PYG@bf=\textbf\def\PYG@tc##1{\textcolor[rgb]{0.00,0.00,0.50}{##1}}}
\expandafter\def\csname PYG@tok@ni\endcsname{\let\PYG@bf=\textbf\def\PYG@tc##1{\textcolor[rgb]{0.84,0.33,0.22}{##1}}}
\expandafter\def\csname PYG@tok@nl\endcsname{\let\PYG@bf=\textbf\def\PYG@tc##1{\textcolor[rgb]{0.00,0.13,0.44}{##1}}}
\expandafter\def\csname PYG@tok@nn\endcsname{\let\PYG@bf=\textbf\def\PYG@tc##1{\textcolor[rgb]{0.05,0.52,0.71}{##1}}}
\expandafter\def\csname PYG@tok@no\endcsname{\def\PYG@tc##1{\textcolor[rgb]{0.38,0.68,0.84}{##1}}}
\expandafter\def\csname PYG@tok@na\endcsname{\def\PYG@tc##1{\textcolor[rgb]{0.25,0.44,0.63}{##1}}}
\expandafter\def\csname PYG@tok@nb\endcsname{\def\PYG@tc##1{\textcolor[rgb]{0.00,0.44,0.13}{##1}}}
\expandafter\def\csname PYG@tok@nc\endcsname{\let\PYG@bf=\textbf\def\PYG@tc##1{\textcolor[rgb]{0.05,0.52,0.71}{##1}}}
\expandafter\def\csname PYG@tok@nd\endcsname{\let\PYG@bf=\textbf\def\PYG@tc##1{\textcolor[rgb]{0.33,0.33,0.33}{##1}}}
\expandafter\def\csname PYG@tok@ne\endcsname{\def\PYG@tc##1{\textcolor[rgb]{0.00,0.44,0.13}{##1}}}
\expandafter\def\csname PYG@tok@nf\endcsname{\def\PYG@tc##1{\textcolor[rgb]{0.02,0.16,0.49}{##1}}}
\expandafter\def\csname PYG@tok@si\endcsname{\let\PYG@it=\textit\def\PYG@tc##1{\textcolor[rgb]{0.44,0.63,0.82}{##1}}}
\expandafter\def\csname PYG@tok@s2\endcsname{\def\PYG@tc##1{\textcolor[rgb]{0.25,0.44,0.63}{##1}}}
\expandafter\def\csname PYG@tok@nt\endcsname{\let\PYG@bf=\textbf\def\PYG@tc##1{\textcolor[rgb]{0.02,0.16,0.45}{##1}}}
\expandafter\def\csname PYG@tok@nv\endcsname{\def\PYG@tc##1{\textcolor[rgb]{0.73,0.38,0.84}{##1}}}
\expandafter\def\csname PYG@tok@s1\endcsname{\def\PYG@tc##1{\textcolor[rgb]{0.25,0.44,0.63}{##1}}}
\expandafter\def\csname PYG@tok@ch\endcsname{\let\PYG@it=\textit\def\PYG@tc##1{\textcolor[rgb]{0.25,0.50,0.56}{##1}}}
\expandafter\def\csname PYG@tok@m\endcsname{\def\PYG@tc##1{\textcolor[rgb]{0.13,0.50,0.31}{##1}}}
\expandafter\def\csname PYG@tok@gp\endcsname{\let\PYG@bf=\textbf\def\PYG@tc##1{\textcolor[rgb]{0.78,0.36,0.04}{##1}}}
\expandafter\def\csname PYG@tok@sh\endcsname{\def\PYG@tc##1{\textcolor[rgb]{0.25,0.44,0.63}{##1}}}
\expandafter\def\csname PYG@tok@ow\endcsname{\let\PYG@bf=\textbf\def\PYG@tc##1{\textcolor[rgb]{0.00,0.44,0.13}{##1}}}
\expandafter\def\csname PYG@tok@sx\endcsname{\def\PYG@tc##1{\textcolor[rgb]{0.78,0.36,0.04}{##1}}}
\expandafter\def\csname PYG@tok@bp\endcsname{\def\PYG@tc##1{\textcolor[rgb]{0.00,0.44,0.13}{##1}}}
\expandafter\def\csname PYG@tok@c1\endcsname{\let\PYG@it=\textit\def\PYG@tc##1{\textcolor[rgb]{0.25,0.50,0.56}{##1}}}
\expandafter\def\csname PYG@tok@o\endcsname{\def\PYG@tc##1{\textcolor[rgb]{0.40,0.40,0.40}{##1}}}
\expandafter\def\csname PYG@tok@kc\endcsname{\let\PYG@bf=\textbf\def\PYG@tc##1{\textcolor[rgb]{0.00,0.44,0.13}{##1}}}
\expandafter\def\csname PYG@tok@c\endcsname{\let\PYG@it=\textit\def\PYG@tc##1{\textcolor[rgb]{0.25,0.50,0.56}{##1}}}
\expandafter\def\csname PYG@tok@mf\endcsname{\def\PYG@tc##1{\textcolor[rgb]{0.13,0.50,0.31}{##1}}}
\expandafter\def\csname PYG@tok@err\endcsname{\def\PYG@bc##1{\setlength{\fboxsep}{0pt}\fcolorbox[rgb]{1.00,0.00,0.00}{1,1,1}{\strut ##1}}}
\expandafter\def\csname PYG@tok@mb\endcsname{\def\PYG@tc##1{\textcolor[rgb]{0.13,0.50,0.31}{##1}}}
\expandafter\def\csname PYG@tok@ss\endcsname{\def\PYG@tc##1{\textcolor[rgb]{0.32,0.47,0.09}{##1}}}
\expandafter\def\csname PYG@tok@sr\endcsname{\def\PYG@tc##1{\textcolor[rgb]{0.14,0.33,0.53}{##1}}}
\expandafter\def\csname PYG@tok@mo\endcsname{\def\PYG@tc##1{\textcolor[rgb]{0.13,0.50,0.31}{##1}}}
\expandafter\def\csname PYG@tok@kd\endcsname{\let\PYG@bf=\textbf\def\PYG@tc##1{\textcolor[rgb]{0.00,0.44,0.13}{##1}}}
\expandafter\def\csname PYG@tok@mi\endcsname{\def\PYG@tc##1{\textcolor[rgb]{0.13,0.50,0.31}{##1}}}
\expandafter\def\csname PYG@tok@kn\endcsname{\let\PYG@bf=\textbf\def\PYG@tc##1{\textcolor[rgb]{0.00,0.44,0.13}{##1}}}
\expandafter\def\csname PYG@tok@cpf\endcsname{\let\PYG@it=\textit\def\PYG@tc##1{\textcolor[rgb]{0.25,0.50,0.56}{##1}}}
\expandafter\def\csname PYG@tok@kr\endcsname{\let\PYG@bf=\textbf\def\PYG@tc##1{\textcolor[rgb]{0.00,0.44,0.13}{##1}}}
\expandafter\def\csname PYG@tok@s\endcsname{\def\PYG@tc##1{\textcolor[rgb]{0.25,0.44,0.63}{##1}}}
\expandafter\def\csname PYG@tok@kp\endcsname{\def\PYG@tc##1{\textcolor[rgb]{0.00,0.44,0.13}{##1}}}
\expandafter\def\csname PYG@tok@w\endcsname{\def\PYG@tc##1{\textcolor[rgb]{0.73,0.73,0.73}{##1}}}
\expandafter\def\csname PYG@tok@kt\endcsname{\def\PYG@tc##1{\textcolor[rgb]{0.56,0.13,0.00}{##1}}}
\expandafter\def\csname PYG@tok@sc\endcsname{\def\PYG@tc##1{\textcolor[rgb]{0.25,0.44,0.63}{##1}}}
\expandafter\def\csname PYG@tok@sb\endcsname{\def\PYG@tc##1{\textcolor[rgb]{0.25,0.44,0.63}{##1}}}
\expandafter\def\csname PYG@tok@k\endcsname{\let\PYG@bf=\textbf\def\PYG@tc##1{\textcolor[rgb]{0.00,0.44,0.13}{##1}}}
\expandafter\def\csname PYG@tok@se\endcsname{\let\PYG@bf=\textbf\def\PYG@tc##1{\textcolor[rgb]{0.25,0.44,0.63}{##1}}}
\expandafter\def\csname PYG@tok@sd\endcsname{\let\PYG@it=\textit\def\PYG@tc##1{\textcolor[rgb]{0.25,0.44,0.63}{##1}}}

\def\PYGZbs{\char`\\}
\def\PYGZus{\char`\_}
\def\PYGZob{\char`\{}
\def\PYGZcb{\char`\}}
\def\PYGZca{\char`\^}
\def\PYGZam{\char`\&}
\def\PYGZlt{\char`\<}
\def\PYGZgt{\char`\>}
\def\PYGZsh{\char`\#}
\def\PYGZpc{\char`\%}
\def\PYGZdl{\char`\$}
\def\PYGZhy{\char`\-}
\def\PYGZsq{\char`\'}
\def\PYGZdq{\char`\"}
\def\PYGZti{\char`\~}
% for compatibility with earlier versions
\def\PYGZat{@}
\def\PYGZlb{[}
\def\PYGZrb{]}
\makeatother

\renewcommand\PYGZsq{\textquotesingle}

\begin{document}

\maketitle
\tableofcontents
\phantomsection\label{index::doc}


Contents:


\chapter{meta-raspberrypi}
\label{readme:welcome-to-meta-raspberrypi-s-documentation}\label{readme:meta-raspberrypi}\label{readme::doc}
Yocto BSP layer for the Raspberry Pi boards - \href{http://www.raspberrypi.org/}{http://www.raspberrypi.org/}.

\href{https://yocto-ci.resin.io/job/meta-raspberrypi1}{}
\href{https://yocto-ci.resin.io/job/meta-raspberrypi2}{}
\href{https://yocto-ci.resin.io/job/meta-raspberrypi3}{}
\href{https://yocto-ci.resin.io/job/meta-raspberrypi4}{}
\href{https://meta-raspberrypi.readthedocs.io/en/latest/?badge=latest}{}
\href{https://matrix.to/\#/\#meta-raspberrypi:cub.icu}{}


\section{Quick links}
\label{readme:quick-links}\begin{itemize}
\item {} 
Git repository web frontend:
\href{https://github.com/agherzan/meta-raspberrypi}{https://github.com/agherzan/meta-raspberrypi}

\item {} 
Mailing list (yocto mailing list): \href{mailto:yocto@yoctoproject.org}{yocto@yoctoproject.org}

\item {} 
Issues management (Github Issues):
\href{https://github.com/agherzan/meta-raspberrypi/issues}{https://github.com/agherzan/meta-raspberrypi/issues}

\item {} 
Documentation: \href{http://meta-raspberrypi.readthedocs.io/en/latest/}{http://meta-raspberrypi.readthedocs.io/en/latest/}

\end{itemize}


\section{Description}
\label{readme:description}
This is the general hardware specific BSP overlay for the RaspberryPi device.

More information can be found at: \href{http://www.raspberrypi.org/}{http://www.raspberrypi.org/} (Official Site)

The core BSP part of meta-raspberrypi should work with different
OpenEmbedded/Yocto distributions and layer stacks, such as:
\begin{itemize}
\item {} 
Distro-less (only with OE-Core).

\item {} 
Angstrom.

\item {} 
Yocto/Poky (main focus of testing).

\end{itemize}


\section{Dependencies}
\label{readme:dependencies}
This layer depends on:
\begin{itemize}
\item {} 
URI: git://git.yoctoproject.org/poky
\begin{itemize}
\item {} 
branch: master

\item {} 
revision: HEAD

\end{itemize}

\item {} 
URI: git://git.openembedded.org/meta-openembedded
\begin{itemize}
\item {} 
layers: meta-oe, meta-multimedia, meta-networking, meta-python

\item {} 
branch: master

\item {} 
revision: HEAD

\end{itemize}

\end{itemize}


\section{Quick Start}
\label{readme:quick-start}\begin{enumerate}
\item {} 
source poky/oe-init-build-env rpi-build

\item {} 
Add this layer to bblayers.conf and the dependencies above

\item {} 
Set MACHINE in local.conf to one of the supported boards

\item {} 
bitbake core-image-base

\item {} 
dd to a SD card the generated sdimg file (use xzcat if rpi-sdimg.xz is used)

\item {} 
Boot your RPI.

\end{enumerate}


\section{Quick Start with kas}
\label{readme:quick-start-with-kas}\begin{enumerate}
\item {} 
Install kas build tool from PyPi (sudo pip3 install kas)

\item {} 
kas build meta-raspberrypi/kas-poky-rpi.yml

\item {} 
dd to a SD card the generated sdimg file (use xzcat if rpi-sdimg.xz is used)

\item {} 
Boot your RPI.

\end{enumerate}

To adjust the build configuration with specific options (I2C, SPI, ...), simply add
a section as follows:

\begin{Verbatim}[commandchars=\\\{\}]
local\PYGZus{}conf\PYGZus{}header:
  rpi\PYGZhy{}specific: \textbar{}
    ENABLE\PYGZus{}I2C = \PYGZdq{}1\PYGZdq{}
    RPI\PYGZus{}EXTRA\PYGZus{}CONFIG = \PYGZdq{}dtoverlay=pi3\PYGZhy{}disable\PYGZhy{}bt\PYGZdq{}
\end{Verbatim}

To configure the machine, you have to update the \code{machine} variable.
And the same for the \code{distro}.

For further information, you can read more at \href{https://kas.readthedocs.io/en/1.0/index.html}{https://kas.readthedocs.io/en/1.0/index.html}


\section{Maintainers}
\label{readme:maintainers}\begin{itemize}
\item {} 
Andrei Gherzan \code{\textless{}andrei at gherzan.ro\textgreater{}}

\end{itemize}


\chapter{Layer Contents}
\label{layer-contents:layer-contents}\label{layer-contents::doc}

\section{Supported Machines}
\label{layer-contents:supported-machines}\begin{itemize}
\item {} 
raspberrypi

\item {} 
raspberrypi0

\item {} 
raspberrypi0-wifi

\item {} 
raspberrypi2

\item {} 
raspberrypi3

\item {} 
raspberrypi3-64 (64 bit kernel \& userspace)

\item {} 
raspberrypi-cm (dummy alias for raspberrypi)

\item {} 
raspberrypi-cm3

\end{itemize}

Note: The raspberrypi3 machines include support for Raspberry Pi 3B+.


\section{Images}
\label{layer-contents:images}\begin{itemize}
\item {} 
rpi-test-image
\begin{itemize}
\item {} 
Image based on core-image-base which includes most of the packages in this
layer and some media samples.

\end{itemize}

\end{itemize}

For other uses it`s recommended to base images on \code{core-image-minimal} or
\code{core-image-base} as appropriate. The old image names (\code{rpi-hwup-image} and
\code{rpi-basic-image}) are deprecated.


\section{WiFi and Bluetooth Firmware}
\label{layer-contents:wifi-and-bluetooth-firmware}
Be aware that the WiFi and Bluetooth firmware for the supported boards
is not available in the base version of \code{linux-firmware} from OE-Core
(poky). The files are added from Raspbian repositories in this layer`s
bbappends to that recipe. All machines define
\code{MACHINE\_EXTRA\_RRECOMMENDS} to include the required wireless firmware;
raspberrypi3 supports 3, 3B, and 3B+ and so include multiple firmware
packages.


\chapter{Optional build configuration}
\label{extra-build-config::doc}\label{extra-build-config:optional-build-configuration}
There are a set of ways in which a user can influence different paramenters of
the build. We list here the ones that are closely related to this BSP or
specific to it. For the rest please check:
\href{http://www.yoctoproject.org/docs/latest/ref-manual/ref-manual.html}{http://www.yoctoproject.org/docs/latest/ref-manual/ref-manual.html}


\section{Compressed deployed files}
\label{extra-build-config:compressed-deployed-files}\begin{enumerate}
\item {} 
Overwrite IMAGE\_FSTYPES in local.conf
\begin{itemize}
\item {} 
\code{IMAGE\_FSTYPES = "tar.bz2 ext3.xz"}

\end{itemize}

\item {} 
Overwrite SDIMG\_ROOTFS\_TYPE in local.conf
\begin{itemize}
\item {} 
\code{SDIMG\_ROOTFS\_TYPE = "ext3.xz"}

\end{itemize}

\end{enumerate}

Accommodate the values above to your own needs (ex: ext3 / ext4).


\section{GPU memory}
\label{extra-build-config:gpu-memory}\begin{itemize}
\item {} 
\code{GPU\_MEM}: GPU memory in megabyte. Sets the memory split between the ARM and
GPU. ARM gets the remaining memory. Min 16. Default 64.

\item {} 
\code{GPU\_MEM\_256}: GPU memory in megabyte for the 256MB Raspberry Pi. Ignored by
the 512MB RP. Overrides gpu\_mem. Max 192. Default not set.

\item {} 
\code{GPU\_MEM\_512}: GPU memory in megabyte for the 512MB Raspberry Pi. Ignored by
the 256MB RP. Overrides gpu\_mem. Max 448. Default not set.

\item {} 
\code{GPU\_MEM\_1024}: GPU memory in megabyte for the 1024MB Raspberry Pi. Ignored by
the 256MB/512MB RP. Overrides gpu\_mem. Max 944. Default not set.

\end{itemize}

See: \href{https://www.raspberrypi.org/documentation/configuration/config-txt/memory}{https://www.raspberrypi.org/documentation/configuration/config-txt/memory.md}


\section{VC4}
\label{extra-build-config:vc4}
By default, each machine uses \code{vc4} for graphics. This will in turn sets mesa as provider for \code{gl} libraries. \code{DISABLE\_VC4GRAPHICS} can be set to \code{1} to disable this behaviour falling back to using \code{userland}. Be aware that \code{userland} has not support for 64-bit arch. If you disable \code{vc4} on a 64-bit Raspberry Pi machine, expect build breakage.


\section{Add purchased license codecs}
\label{extra-build-config:add-purchased-license-codecs}
To add you own licenses use variables \code{KEY\_DECODE\_MPG2} and \code{KEY\_DECODE\_WVC1} in
local.conf. Example:

\begin{Verbatim}[commandchars=\\\{\}]
\PYG{n}{KEY\PYGZus{}DECODE\PYGZus{}MPG2} \PYG{o}{=} \PYG{l+s+s2}{\PYGZdq{}}\PYG{l+s+s2}{12345678}\PYG{l+s+s2}{\PYGZdq{}}
\PYG{n}{KEY\PYGZus{}DECODE\PYGZus{}WVC1} \PYG{o}{=} \PYG{l+s+s2}{\PYGZdq{}}\PYG{l+s+s2}{12345678}\PYG{l+s+s2}{\PYGZdq{}}
\end{Verbatim}

You can supply more licenses separated by comma. Example:

\begin{Verbatim}[commandchars=\\\{\}]
\PYG{n}{KEY\PYGZus{}DECODE\PYGZus{}WVC1} \PYG{o}{=} \PYG{l+s+s2}{\PYGZdq{}}\PYG{l+s+s2}{0x12345678,0xabcdabcd,0x87654321}\PYG{l+s+s2}{\PYGZdq{}}
\end{Verbatim}

See: \href{https://www.raspberrypi.org/documentation/configuration/config-txt/codeclicence}{https://www.raspberrypi.org/documentation/configuration/config-txt/codeclicence.md}


\section{Disable overscan}
\label{extra-build-config:disable-overscan}
By default the GPU adds a black border around the video output to compensate for
TVs which cut off part of the image. To disable this set this variable in
local.conf:

\begin{Verbatim}[commandchars=\\\{\}]
\PYG{n}{DISABLE\PYGZus{}OVERSCAN} \PYG{o}{=} \PYG{l+s+s2}{\PYGZdq{}}\PYG{l+s+s2}{1}\PYG{l+s+s2}{\PYGZdq{}}
\end{Verbatim}


\section{Disable splash screen}
\label{extra-build-config:disable-splash-screen}
By default a rainbow splash screen is shown after the GPU firmware is loaded.
To disable this set this variable in local.conf:

\begin{Verbatim}[commandchars=\\\{\}]
\PYG{n}{DISABLE\PYGZus{}SPLASH} \PYG{o}{=} \PYG{l+s+s2}{\PYGZdq{}}\PYG{l+s+s2}{1}\PYG{l+s+s2}{\PYGZdq{}}
\end{Verbatim}


\section{Boot delay}
\label{extra-build-config:boot-delay}
The Raspberry Pi waits a number of seconds after loading the GPU firmware and
before loading the kernel. By default it is one second. This is useful if your
SD card needs a while to get ready before Linux is able to boot from it.
To remove (or adjust) this delay set these variables in local.conf:

\begin{Verbatim}[commandchars=\\\{\}]
\PYG{n}{BOOT\PYGZus{}DELAY} \PYG{o}{=} \PYG{l+s+s2}{\PYGZdq{}}\PYG{l+s+s2}{0}\PYG{l+s+s2}{\PYGZdq{}}
\PYG{n}{BOOT\PYGZus{}DELAY\PYGZus{}MS} \PYG{o}{=} \PYG{l+s+s2}{\PYGZdq{}}\PYG{l+s+s2}{0}\PYG{l+s+s2}{\PYGZdq{}}
\end{Verbatim}


\section{Set overclocking options}
\label{extra-build-config:set-overclocking-options}
The Raspberry Pi can be overclocked. As of now overclocking up to the ``Turbo
Mode`` is officially supported by the raspbery and does not void warranty. Check
the config.txt for a detailed description of options and modes. The following
variables are supported in local.conf: \code{ARM\_FREQ}, \code{GPU\_FREQ}, \code{CORE\_FREQ},
\code{SDRAM\_FREQ} and \code{OVER\_VOLTAGE}.

Example official settings for Turbo Mode in Raspberry Pi 2:

\begin{Verbatim}[commandchars=\\\{\}]
\PYG{n}{ARM\PYGZus{}FREQ} \PYG{o}{=} \PYG{l+s+s2}{\PYGZdq{}}\PYG{l+s+s2}{1000}\PYG{l+s+s2}{\PYGZdq{}}
\PYG{n}{CORE\PYGZus{}FREQ} \PYG{o}{=} \PYG{l+s+s2}{\PYGZdq{}}\PYG{l+s+s2}{500}\PYG{l+s+s2}{\PYGZdq{}}
\PYG{n}{SDRAM\PYGZus{}FREQ} \PYG{o}{=} \PYG{l+s+s2}{\PYGZdq{}}\PYG{l+s+s2}{500}\PYG{l+s+s2}{\PYGZdq{}}
\PYG{n}{OVER\PYGZus{}VOLTAGE} \PYG{o}{=} \PYG{l+s+s2}{\PYGZdq{}}\PYG{l+s+s2}{6}\PYG{l+s+s2}{\PYGZdq{}}
\end{Verbatim}

See: \href{https://www.raspberrypi.org/documentation/configuration/config-txt/overclocking}{https://www.raspberrypi.org/documentation/configuration/config-txt/overclocking.md}


\section{HDMI and composite video options}
\label{extra-build-config:hdmi-and-composite-video-options}
The Raspberry Pi can output video over HDMI or SDTV composite (the RCA connector).
By default the video mode for these is autodetected on boot: the HDMI mode is
selected according to the connected monitor`s EDID information and the composite
mode is defaulted to NTSC using a 4:3 aspect ratio. Check the config.txt for a
detailed description of options and modes. The following variables are supported in
local.conf: \code{HDMI\_FORCE\_HOTPLUG}, \code{HDMI\_DRIVE}, \code{HDMI\_GROUP}, \code{HDMI\_MODE},
\code{CONFIG\_HDMI\_BOOST}, \code{SDTV\_MODE}, \code{SDTV\_ASPECT} and \code{DISPLAY\_ROTATE}.

Example to force HDMI output to 720p in CEA mode:

\begin{Verbatim}[commandchars=\\\{\}]
\PYG{n}{HDMI\PYGZus{}GROUP} \PYG{o}{=} \PYG{l+s+s2}{\PYGZdq{}}\PYG{l+s+s2}{1}\PYG{l+s+s2}{\PYGZdq{}}
\PYG{n}{HDMI\PYGZus{}MODE} \PYG{o}{=} \PYG{l+s+s2}{\PYGZdq{}}\PYG{l+s+s2}{4}\PYG{l+s+s2}{\PYGZdq{}}
\end{Verbatim}

See: \href{https://www.raspberrypi.org/documentation/configuration/config-txt/video}{https://www.raspberrypi.org/documentation/configuration/config-txt/video.md}


\section{Video camera support with V4L2 drivers}
\label{extra-build-config:video-camera-support-with-v4l2-drivers}
Set this variable to enable support for the video camera (Linux 3.12.4+
required):

\begin{Verbatim}[commandchars=\\\{\}]
\PYG{n}{VIDEO\PYGZus{}CAMERA} \PYG{o}{=} \PYG{l+s+s2}{\PYGZdq{}}\PYG{l+s+s2}{1}\PYG{l+s+s2}{\PYGZdq{}}
\end{Verbatim}


\section{Enable offline compositing support}
\label{extra-build-config:enable-offline-compositing-support}
Set this variable to enable support for dispmanx offline compositing:

\begin{Verbatim}[commandchars=\\\{\}]
\PYG{n}{DISPMANX\PYGZus{}OFFLINE} \PYG{o}{=} \PYG{l+s+s2}{\PYGZdq{}}\PYG{l+s+s2}{1}\PYG{l+s+s2}{\PYGZdq{}}
\end{Verbatim}

This will enable the firmware to fall back to off-line compositing of Dispmanx
elements. Normally the compositing is done on-line, during scanout, but cannot
handle too many elements. With off-line enabled, an off-screen buffer is
allocated for compositing. When scene complexity (number and sizes
of elements) is high, compositing will happen off-line into the buffer.

Heavily recommended for Wayland/Weston.

See: \href{http://wayland.freedesktop.org/raspberrypi.html}{http://wayland.freedesktop.org/raspberrypi.html}


\section{Enable kgdb over console support}
\label{extra-build-config:enable-kgdb-over-console-support}
To add the kdbg over console (kgdboc) parameter to the kernel command line, set
this variable in local.conf:

\begin{Verbatim}[commandchars=\\\{\}]
\PYG{n}{ENABLE\PYGZus{}KGDB} \PYG{o}{=} \PYG{l+s+s2}{\PYGZdq{}}\PYG{l+s+s2}{1}\PYG{l+s+s2}{\PYGZdq{}}
\end{Verbatim}


\section{Disable rpi boot logo}
\label{extra-build-config:disable-rpi-boot-logo}
To disable rpi boot logo, set this variable in local.conf:

\begin{Verbatim}[commandchars=\\\{\}]
\PYG{n}{DISABLE\PYGZus{}RPI\PYGZus{}BOOT\PYGZus{}LOGO} \PYG{o}{=} \PYG{l+s+s2}{\PYGZdq{}}\PYG{l+s+s2}{1}\PYG{l+s+s2}{\PYGZdq{}}
\end{Verbatim}


\section{Boot to U-Boot}
\label{extra-build-config:boot-to-u-boot}
To have u-boot load kernel image, set in your local.conf:

\begin{Verbatim}[commandchars=\\\{\}]
\PYG{n}{RPI\PYGZus{}USE\PYGZus{}U\PYGZus{}BOOT} \PYG{o}{=} \PYG{l+s+s2}{\PYGZdq{}}\PYG{l+s+s2}{1}\PYG{l+s+s2}{\PYGZdq{}}
\end{Verbatim}

This will select the appropriate image format for use with u-boot automatically.
For further customisation the KERNEL\_IMAGETYPE and KERNEL\_BOOTCMD variables can
be overridden to select the exact kernel image type (eg. zImage) and u-boot
command (eg. bootz) to be used.


\section{Image with Initramfs}
\label{extra-build-config:image-with-initramfs}
To build an initramfs image:
\begin{itemize}
\item {} 
Set this 3 kernel variables (in kernel`s do\_configure\_prepend in linux-raspberrypi.inc after the line kernel\_configure\_variable LOCALVERSION ````````
)
\begin{itemize}
\item {} 
kernel\_configure\_variable BLK\_DEV\_INITRD y

\item {} 
kernel\_configure\_variable INITRAMFS\_SOURCE ````

\item {} 
kernel\_configure\_variable RD\_GZIP y

\end{itemize}

\item {} 
Set the yocto variables (e.g. in local.conf)
\begin{itemize}
\item {} 
\code{INITRAMFS\_IMAGE = "\textless{}name for your initramfs image\textgreater{}"}

\item {} 
\code{INITRAMFS\_IMAGE\_BUNDLE = "1"}

\item {} 
\code{BOOT\_SPACE = "1073741"}

\item {} 
\code{INITRAMFS\_MAXSIZE = "315400"}

\item {} 
\code{IMAGE\_FSTYPES\_pn-\$\{INITRAMFS\_IMAGE\} = "\$\{INITRAMFS\_FSTYPES\}"}

\end{itemize}

\end{itemize}


\section{Enable SPI bus}
\label{extra-build-config:enable-spi-bus}
When using device tree kernels, set this variable to enable the SPI bus:

\begin{Verbatim}[commandchars=\\\{\}]
\PYG{n}{ENABLE\PYGZus{}SPI\PYGZus{}BUS} \PYG{o}{=} \PYG{l+s+s2}{\PYGZdq{}}\PYG{l+s+s2}{1}\PYG{l+s+s2}{\PYGZdq{}}
\end{Verbatim}


\section{Enable I2C}
\label{extra-build-config:enable-i2c}
When using device tree kernels, set this variable to enable I2C:

\begin{Verbatim}[commandchars=\\\{\}]
\PYG{n}{ENABLE\PYGZus{}I2C} \PYG{o}{=} \PYG{l+s+s2}{\PYGZdq{}}\PYG{l+s+s2}{1}\PYG{l+s+s2}{\PYGZdq{}}
\end{Verbatim}

Furthermore, to auto-load I2C kernel modules set:

\begin{Verbatim}[commandchars=\\\{\}]
\PYG{n}{KERNEL\PYGZus{}MODULE\PYGZus{}AUTOLOAD\PYGZus{}rpi} \PYG{o}{+}\PYG{o}{=} \PYG{l+s+s2}{\PYGZdq{}}\PYG{l+s+s2}{i2c\PYGZhy{}dev i2c\PYGZhy{}bcm2708}\PYG{l+s+s2}{\PYGZdq{}}
\end{Verbatim}


\section{Enable PiTFT support}
\label{extra-build-config:enable-pitft-support}
Basic support for using PiTFT screens can be enabled by adding below in
local.conf:
\begin{itemize}
\item {} 
\code{MACHINE\_FEATURES += "pitft"}
\begin{itemize}
\item {} 
This will enable SPI bus and i2c device-trees, it will also setup
framebuffer for console and x server on PiTFT.

\end{itemize}

\end{itemize}

NOTE: To get this working the overlay for the PiTFT model must be build, added
and specified as well (dtoverlay= in config.txt).

Below is a list of currently supported PiTFT models in meta-raspberrypi, the
modelname should be added as a MACHINE\_FEATURES in local.conf like below:

\begin{Verbatim}[commandchars=\\\{\}]
\PYG{n}{MACHINE\PYGZus{}FEATURES} \PYG{o}{+}\PYG{o}{=} \PYG{l+s+s2}{\PYGZdq{}}\PYG{l+s+s2}{pitft \PYGZlt{}modelname\PYGZgt{}}\PYG{l+s+s2}{\PYGZdq{}}
\end{Verbatim}

List of currently supported models:
\begin{itemize}
\item {} 
pitft22

\item {} 
pitft28r

\item {} 
pitft28c

\item {} 
pitft35r

\end{itemize}


\section{Misc. display}
\label{extra-build-config:misc-display}
If you would like to use the Waveshare ``C`` 1024×600, 7 inch Capacitive Touch
Screen LCD, HDMI interface (\href{http://www.waveshare.com/7inch-HDMI-LCD-C.htm}{http://www.waveshare.com/7inch-HDMI-LCD-C.htm}) Rev
2.1, please set the following in your local.conf:

\begin{Verbatim}[commandchars=\\\{\}]
\PYG{n}{WAVESHARE\PYGZus{}1024X600\PYGZus{}C\PYGZus{}2\PYGZus{}1} \PYG{o}{=} \PYG{l+s+s2}{\PYGZdq{}}\PYG{l+s+s2}{1}\PYG{l+s+s2}{\PYGZdq{}}
\end{Verbatim}


\section{Enable UART}
\label{extra-build-config:enable-uart}
RaspberryPi 0, 1, 2 and CM will have UART console enabled by default.

RaspberryPi 0 WiFi and 3 does not have the UART enabled by default because this
needs a fixed core frequency and enable\_uart wil set it to the minimum. Certain
operations - 60fps h264 decode, high quality deinterlace - which aren`t
performed on the ARM may be affected, and we wouldn`t want to do that to users
who don`t want to use the serial port. Users who want serial console support on
RaspberryPi 0 Wifi or 3 will have to explicitly set in local.conf:

\begin{Verbatim}[commandchars=\\\{\}]
\PYG{n}{ENABLE\PYGZus{}UART} \PYG{o}{=} \PYG{l+s+s2}{\PYGZdq{}}\PYG{l+s+s2}{1}\PYG{l+s+s2}{\PYGZdq{}}
\end{Verbatim}

Ref.:
\begin{itemize}
\item {} 
\href{https://github.com/raspberrypi/firmware/issues/553}{https://github.com/raspberrypi/firmware/issues/553}

\item {} 
\href{https://github.com/RPi-Distro/repo/issues/22}{https://github.com/RPi-Distro/repo/issues/22}

\end{itemize}


\section{Enable USB Peripheral (Gadget) support}
\label{extra-build-config:enable-usb-peripheral-gadget-support}
The standard USB driver only supports host mode operations.  Users who
want to use gadget modules like g\_ether should set the following in
local.conf:

\begin{Verbatim}[commandchars=\\\{\}]
\PYG{n}{ENABLE\PYGZus{}DWC2\PYGZus{}PERIPHERAL} \PYG{o}{=} \PYG{l+s+s2}{\PYGZdq{}}\PYG{l+s+s2}{1}\PYG{l+s+s2}{\PYGZdq{}}
\end{Verbatim}


\section{Enable Openlabs 802.15.4 radio module}
\label{extra-build-config:enable-openlabs-802-15-4-radio-module}
When using device tree kernels, set this variable to enable the 802.15.4 hat:

\begin{Verbatim}[commandchars=\\\{\}]
\PYG{n}{ENABLE\PYGZus{}AT86RF} \PYG{o}{=} \PYG{l+s+s2}{\PYGZdq{}}\PYG{l+s+s2}{1}\PYG{l+s+s2}{\PYGZdq{}}
\end{Verbatim}

See: \href{https://openlabs.co/OSHW/Raspberry-Pi-802.15.4-radio}{https://openlabs.co/OSHW/Raspberry-Pi-802.15.4-radio}


\section{Enable CAN with Pican2}
\label{extra-build-config:enable-can-with-pican2}
In order to use Pican2 CAN module, set the following variables:

\begin{Verbatim}[commandchars=\\\{\}]
\PYG{n}{ENABLE\PYGZus{}SPI\PYGZus{}BUS} \PYG{o}{=} \PYG{l+s+s2}{\PYGZdq{}}\PYG{l+s+s2}{1}\PYG{l+s+s2}{\PYGZdq{}}
\PYG{n}{ENABLE\PYGZus{}CAN} \PYG{o}{=} \PYG{l+s+s2}{\PYGZdq{}}\PYG{l+s+s2}{1}\PYG{l+s+s2}{\PYGZdq{}}
\end{Verbatim}

See: \href{http://skpang.co.uk/catalog/pican2-canbus-board-for-raspberry-pi-23-p-1475.html}{http://skpang.co.uk/catalog/pican2-canbus-board-for-raspberry-pi-23-p-1475.html}


\section{Manual additions to config.txt}
\label{extra-build-config:manual-additions-to-config-txt}
The \code{RPI\_EXTRA\_CONFIG} variable can be used to manually add additional lines to
the \code{config.txt} file if there is not a specific option above for the
configuration you need. To add multiple lines you must include \code{\textbackslash{}n} separators.
If double-quotes are needed in the lines you are adding you can use single
quotes around the whole string.

For example, to add a comment containing a double-quote and a configuration
option:

\begin{Verbatim}[commandchars=\\\{\}]
\PYG{n}{RPI\PYGZus{}EXTRA\PYGZus{}CONFIG} \PYG{o}{=} \PYG{l+s+s1}{\PYGZsq{}}\PYG{l+s+s1}{ }\PYG{l+s+se}{\PYGZbs{}n}\PYG{l+s+s1}{ }\PYG{l+s+se}{\PYGZbs{}}
\PYG{l+s+s1}{    \PYGZsh{} Raspberry Pi 7}\PYG{l+s+se}{\PYGZbs{}\PYGZdq{}}\PYG{l+s+s1}{ display/touch screen }\PYG{l+s+se}{\PYGZbs{}n}\PYG{l+s+s1}{ }\PYG{l+s+se}{\PYGZbs{}}
\PYG{l+s+s1}{    lcd\PYGZus{}rotate=2 }\PYG{l+s+se}{\PYGZbs{}n}\PYG{l+s+s1}{ }\PYG{l+s+se}{\PYGZbs{}}
\PYG{l+s+s1}{    }\PYG{l+s+s1}{\PYGZsq{}}
\end{Verbatim}


\chapter{Extra apps}
\label{extra-apps:extra-apps}\label{extra-apps::doc}

\section{omxplayer}
\label{extra-apps:omxplayer}
omxplayer depends on libav which has a commercial license. So in order to be
able to compile omxplayer you will need to whiteflag the commercial
license in your local.conf:

\begin{Verbatim}[commandchars=\\\{\}]
\PYG{n}{LICENSE\PYGZus{}FLAGS\PYGZus{}WHITELIST} \PYG{o}{=} \PYG{l+s+s2}{\PYGZdq{}}\PYG{l+s+s2}{commercial}\PYG{l+s+s2}{\PYGZdq{}}
\end{Verbatim}


\chapter{Contributing}
\label{contributing:contributing}\label{contributing::doc}

\section{Mailing list}
\label{contributing:mailing-list}
The main communication tool in use is the Yocto Project mailing list:
\begin{itemize}
\item {} 
\href{mailto:yocto@yoctoproject.org}{yocto@yoctoproject.org}

\item {} 
\href{https://lists.yoctoproject.org/listinfo/yocto}{https://lists.yoctoproject.org/listinfo/yocto}

\end{itemize}

Feel free to ask any kind of questions but please always prepend your email
subject with \code{{[}meta-raspberrypi{]}} as this is the global \emph{Yocto} mailing
list and not a dedicated \emph{meta-raspberrypi} mailing list.


\section{Formatting patches}
\label{contributing:formatting-patches}
First and foremost, all of the contributions to the layer must be compliant
with the standard openembedded patch guidelines:
\begin{itemize}
\item {} 
\href{http://www.openembedded.org/wiki/Commit\_Patch\_Message\_Guidelines}{http://www.openembedded.org/wiki/Commit\_Patch\_Message\_Guidelines}

\end{itemize}

In summary, your commit log messages should be formatted as follows:

\begin{Verbatim}[commandchars=\\\{\}]
\PYGZlt{}layer\PYGZhy{}component\PYGZgt{}: \PYGZlt{}short log/statement of what needed to be changed\PYGZgt{}

(Optional pointers to external resources, such as defect tracking)

The intent of your change.

(Optional: if it\PYGZsq{}s not clear from above, how your change resolves
the issues in the first part)

Signed\PYGZhy{}off\PYGZhy{}by: Your Name \PYGZlt{}yourname@youremail.com\PYGZgt{}
\end{Verbatim}

The \code{\textless{}layer-component\textgreater{}} is the layer component name that your changes affect.
It is important that you choose it correctly. A simple guide for selecting a
a good component name is the following:
\begin{itemize}
\item {} 
For changes that affect \emph{layer recipes}, please just use the \textbf{base names}
of the affected recipes, separated by commas (\code{,}), as the component name.
For example: use \code{omxplayer} instead of \code{omxplayer\_git.bb}. If you are
adding new recipe(s), just use the new recipe(s) base name(s). An example
for changes to multiple recipes would be \code{userland,vc-graphics,wayland}.

\item {} 
For changes that affect the \emph{layer documentation}, please just use \code{docs}
as the component name.

\item {} 
For changes that affect \emph{other files}, i.e. under the \code{conf} directory,
please use the full path as the component name, e.g. \code{conf/layer.conf}.

\item {} 
For changes that affect the \emph{layer itself} and do not fall into any of
the above cases, please use \code{meta-raspberrypi} as the component name.

\end{itemize}

A full example of a suitable commit log message is below:

\begin{Verbatim}[commandchars=\\\{\}]
foobar: Adjusted the foo setting in bar

When using foobar on systems with less than a gigabyte of RAM common
usage patterns often result in an Out\PYGZhy{}of\PYGZhy{}memory condition causing
slowdowns and unexpected application termination.

Low\PYGZhy{}memory systems should continue to function without running into
memory\PYGZhy{}starvation conditions with minimal cost to systems with more
available memory.  High\PYGZhy{}memory systems will be less able to use the
full extent of the system, a dynamically tunable option may be best,
long\PYGZhy{}term.

The foo setting in bar was decreased from X to X\PYGZhy{}50\PYGZpc{} in order to
ensure we don\PYGZsq{}t exhaust all system memory with foobar threads.

Signed\PYGZhy{}off\PYGZhy{}by: Joe Developer \PYGZlt{}joe.developer@example.com\PYGZgt{}
\end{Verbatim}

A common issue during patch reviewing is commit log formatting, please review
the above formatting guidelines carefully before sending your patches.


\section{Sending patches}
\label{contributing:sending-patches}
The preferred method to contribute to this project is to send pull
requests to the GitHub mirror of the layer:
\begin{itemize}
\item {} 
\href{https://github.com/agherzan/meta-raspberrypi}{https://github.com/agherzan/meta-raspberrypi}

\end{itemize}

\textbf{In addition}, you may send patches for review to the above specified
mailing list. In this case, when creating patches using \code{git} please make
sure to use the following formatting for the message subject:

\begin{Verbatim}[commandchars=\\\{\}]
git format\PYGZhy{}patch \PYGZhy{}s \PYGZhy{}\PYGZhy{}subject\PYGZhy{}prefix=\PYGZsq{}meta\PYGZhy{}raspberrypi][PATCH\PYGZsq{} origin
\end{Verbatim}

Then, for sending patches to the mailing list, you may use this command:

\begin{Verbatim}[commandchars=\\\{\}]
git send\PYGZhy{}email \PYGZhy{}\PYGZhy{}to yocto@yoctoproject.org \PYGZlt{}generated patch\PYGZgt{}
\end{Verbatim}


\section{GitHub issues}
\label{contributing:github-issues}
In order to manage and track the layer issues more efficiently, the
GitHub issues facility is used by this project:
\begin{itemize}
\item {} 
\href{https://github.com/agherzan/meta-raspberrypi/issues}{https://github.com/agherzan/meta-raspberrypi/issues}

\end{itemize}

If you submit patches that have a GitHub issue associated, please make sure to
use standard GitHub keywords, e.g. \code{closes}, \code{resolves} or \code{fixes}, before the
``Signed-off-by`` tag to close the relevant issues automatically:

\begin{Verbatim}[commandchars=\\\{\}]
foobar: Adjusted the foo setting in bar

Fixes: \PYGZsh{}324

Signed\PYGZhy{}off\PYGZhy{}by: Joe Developer \PYGZlt{}joe.developer@example.com\PYGZgt{}
\end{Verbatim}

More information on the available GitHub close keywords can be found here:
\begin{itemize}
\item {} 
\href{https://help.github.com/articles/closing-issues-using-keywords}{https://help.github.com/articles/closing-issues-using-keywords}

\end{itemize}


\chapter{Indices and tables}
\label{index:indices-and-tables}\begin{itemize}
\item {} 
\DUspan{xref,std,std-ref}{genindex}

\item {} 
\DUspan{xref,std,std-ref}{modindex}

\item {} 
\DUspan{xref,std,std-ref}{search}

\end{itemize}



\renewcommand{\indexname}{Index}
\printindex
\end{document}
